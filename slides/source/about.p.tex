%%
%% Automatically generated file from DocOnce source
%% (https://github.com/hplgit/doconce/)
%%
%%
% #ifdef PTEX2TEX_EXPLANATION
%%
%% The file follows the ptex2tex extended LaTeX format, see
%% ptex2tex: http://code.google.com/p/ptex2tex/
%%
%% Run
%%      ptex2tex myfile
%% or
%%      doconce ptex2tex myfile
%%
%% to turn myfile.p.tex into an ordinary LaTeX file myfile.tex.
%% (The ptex2tex program: http://code.google.com/p/ptex2tex)
%% Many preprocess options can be added to ptex2tex or doconce ptex2tex
%%
%%      ptex2tex -DMINTED myfile
%%      doconce ptex2tex myfile envir=minted
%%
%% ptex2tex will typeset code environments according to a global or local
%% .ptex2tex.cfg configure file. doconce ptex2tex will typeset code
%% according to options on the command line (just type doconce ptex2tex to
%% see examples). If doconce ptex2tex has envir=minted, it enables the
%% minted style without needing -DMINTED.
% #endif

% #define PREAMBLE

% #ifdef PREAMBLE
%-------------------- begin preamble ----------------------

\documentclass[%
oneside,                 % oneside: electronic viewing, twoside: printing
final,                   % draft: marks overfull hboxes, figures with paths
10pt]{article}

\listfiles               %  print all files needed to compile this document

\usepackage{relsize,makeidx,color,setspace,amsmath,amsfonts,amssymb}
\usepackage[table]{xcolor}
\usepackage{bm,ltablex,microtype}

\usepackage[pdftex]{graphicx}

\usepackage[T1]{fontenc}
%\usepackage[latin1]{inputenc}
\usepackage{ucs}
\usepackage[utf8x]{inputenc}

\usepackage{lmodern}         % Latin Modern fonts derived from Computer Modern

% Hyperlinks in PDF:
\definecolor{linkcolor}{rgb}{0,0,0.4}
\usepackage{hyperref}
\hypersetup{
    breaklinks=true,
    colorlinks=true,
    linkcolor=linkcolor,
    urlcolor=linkcolor,
    citecolor=black,
    filecolor=black,
    %filecolor=blue,
    pdfmenubar=true,
    pdftoolbar=true,
    bookmarksdepth=3   % Uncomment (and tweak) for PDF bookmarks with more levels than the TOC
    }
%\hyperbaseurl{}   % hyperlinks are relative to this root

\setcounter{tocdepth}{2}  % levels in table of contents

% Tricks for having figures close to where they are defined:
% 1. define less restrictive rules for where to put figures
\setcounter{topnumber}{2}
\setcounter{bottomnumber}{2}
\setcounter{totalnumber}{4}
\renewcommand{\topfraction}{0.95}
\renewcommand{\bottomfraction}{0.95}
\renewcommand{\textfraction}{0}
\renewcommand{\floatpagefraction}{0.75}
% floatpagefraction must always be less than topfraction!
% 2. ensure all figures are flushed before next section
\usepackage[section]{placeins}
% 3. enable begin{figure}[H] (often leads to ugly pagebreaks)
%\usepackage{float}\restylefloat{figure}

\usepackage[framemethod=TikZ]{mdframed}

% --- begin definitions of admonition environments ---

% --- end of definitions of admonition environments ---

% prevent orhpans and widows
\clubpenalty = 10000
\widowpenalty = 10000

% --- end of standard preamble for documents ---


% insert custom LaTeX commands...

\raggedbottom
\makeindex
\usepackage[totoc]{idxlayout}   % for index in the toc
\usepackage[nottoc]{tocbibind}  % for references/bibliography in the toc

%-------------------- end preamble ----------------------

\begin{document}

% matching end for #ifdef PREAMBLE
% #endif

\newcommand{\exercisesection}[1]{\subsection*{#1}}

\input{newcommands}

% ------------------- main content ----------------------



% ----------------- title -------------------------

\thispagestyle{empty}

\begin{center}
{\LARGE\bf
\begin{spacing}{1.25}
Kort om kursene INF1100 og MAT-INF1100L
\end{spacing}
}
\end{center}

% ----------------- author(s) -------------------------

\begin{center}
{\bf Hans Petter Langtangen${}^{1, 2}$} \\ [0mm]
\end{center}

\begin{center}
% List of all institutions:
\centerline{{\small ${}^1$Simula Research Laboratory}}
\centerline{{\small ${}^2$University of Oslo, Dept.~of Informatics}}
\end{center}
    
% ----------------- end author(s) -------------------------

% --- begin date ---
\begin{center}
Aug 8, 2017
\end{center}
% --- end date ---

\vspace{1cm}


% !split
\subsection*{INF1100 er en første introduksjon til å programmere datamaskiner}


% --- begin paragraph admon ---
\paragraph{}
\begin{itemize}
  \item Programmering er \emph{svært} viktig i industri og forskning!

  \item Programmering vil bli brukt i veldig mange senere emner - derfor er INF1100/MAT-INF1100L svært sentrale kurs

  \item Hvorfor?\\
    Programmeringen gjør matematikken mye mer anvendbar

  \item Tre perspektiver på matematikk i høst:
\begin{itemize}

    \item tradisjonell kalkulus (MAT1100/MAT1001)

    \item numerisk (datamaskinvennlig) matematikk (MAT-INF1100)

    \item programmering av numerisk matematikk (INF1100)
\end{itemize}

\noindent
\end{itemize}

\noindent
% --- end paragraph admon ---



% !split
\subsection*{MAT-INF1100L = INF1100 uke 1-6 + MAT-INF1100}


% --- begin paragraph admon ---
\paragraph{}
\begin{itemize}
 \item Fullstendig sammenfallende undervisning og obliger med INF1100:
\begin{itemize}

   \item forelesninger: uke 34-39

   \item gruppeøvelser (obliger): uke 35-40

\end{itemize}

\noindent
 \item Samme midtveiseksamen som INF1100

 \item Fullstendig sammenfallende undervisning med MAT-INF1100 etter det
\end{itemize}

\noindent
% --- end paragraph admon ---



% !split
\subsection*{All informasjon og alle beskjeder ligger på nettsidene}

\begin{itemize}
 \item INF1100:
   \href{{http://www.uio.no/studier/emner/matnat/ifi/INF1100/h14}}{\nolinkurl{http://www.uio.no/studier/emner/matnat/ifi/INF1100/h14}}

 \item MAT-INF1100L:
   \href{{http://www.uio.no/studier/emner/matnat/math/MAT-INF1100L-h14}}{\nolinkurl{http://www.uio.no/studier/emner/matnat/math/MAT-INF1100L-h14}}

 \item Se spesielt \href{{https://www.uio.no/studier/emner/matnat/ifi/INF1100/h14/ressurser/undervisningsplan.html}}{INF1100 undervisningsplan} for info om hva som skjer hver uke
\end{itemize}

\noindent
% !split
\subsection*{Undervisningen består av øvelser og forelesninger}


% --- begin paragraph admon ---
\paragraph{}
% !bpop
\begin{itemize}
  \item Plenumsundervisning tirsdager og torsdager 14.15-16.00 i Sophus Lies auditorium

  \item 1. time: oppgaver fra forrige forelesningstime løses i plenum

  \item 2. time: forelesning av nytt stoff

  \item 2 t oppgaveløsning på terminalstue i mindre grupper der du kan få individuell veiledning

  \item Delta på \emph{alle} undervisningstimene!
\end{itemize}

\noindent
% !epop
% --- end paragraph admon ---



% !split
\subsection*{Undervisningsmateriell}

% !bslidecell 00 0.65

% --- begin paragraph admon ---
\paragraph{}
\begin{itemize}
  \item Lærebok skrevet spesielt for INF1100

  \item Oppgavene foreligger som "PDF fil": "

  \item 2. time: forelesning av nytt stoff

  \item 2 t oppgaveløsning på terminalstue i mindre grupper der du kan få individuell veiledning

  \item Delta på \emph{alle} undervisningstimene!
\end{itemize}

\noindent
% --- end paragraph admon ---


% !eslidecell

% !bslidecell 01 0.35
% FIGURE: [http://hplgit.github.io/scipro-primer/figs/Primer4th_pic.jpg, width=350 frac=0.6]
% !eslidecell

% !split
\subsection*{Det kreves innlevering av 3-5 obligatoriske oppgaver hver uke}


% --- begin paragraph admon ---
\paragraph{}
\begin{itemize}
  \item ``Løp 1'': Mange små obligatoriske oppgaver
\begin{itemize}

    \item 3-5 obligatoriske oppgaver hver uke\\
      (vurderes til bestått eller ikke bestått)

    \item De fleste oppgavene teller 1 poeng

    \item Krav INF1100: 15 (av 23) p før uke 41, + 20 (av 37) p før 1.~des.

    \item Krav MAT-INF1100L: 18 (av 23) p fra oppgavene i uke 35-39, men det blir gitt
      ekstraoppgaver etter midtveiseksamen

\end{itemize}

\noindent
  \item ``Løp 2'': Færre, men større obligatoriske oppgaver
\begin{itemize}

    \item Passer for dere med god programmeringserfaring

\end{itemize}

\noindent
  \item Eksamen:
\begin{itemize}

    \item Midtveiseksamen i uke 41 - teller 25\% av karakteren

    \item Avsluttende eksamen - teller 75\% av karakteren
\end{itemize}

\noindent
\end{itemize}

\noindent
% --- end paragraph admon ---



% !split
\subsection*{Hvordan du må jobbe}

% !bpop

% --- begin paragraph admon ---
\paragraph{}
\begin{itemize}
  \item Foran hver forelesning må du ha lest ukens kapittel i læreboken

  \item Foran hver oppgaveløsning i plenum må du selv ha forsøkt å løse oppgavene (les kapittelet først!)

  \item Etterarbeid oppgavene når du har sett løsning i plenum

  \item Nå er du klar for ukens obliger: Du kan gjøre dem på terminalstue
    under veiledning
\end{itemize}

\noindent
% --- end paragraph admon ---




% --- begin paragraph admon ---
\paragraph{Merk:}
\begin{itemize}
  \item Spesielt forelesningene går frem mye fortere enn klasseromsundervisningen i videregående skole

  \item Undervisningen forutsetter at du er forberedt og at du kan forrige ukes temaer
\end{itemize}

\noindent
% --- end paragraph admon ---


% !epop

% !split
\subsection*{Du må lære programmering ved å programmere mye}


% --- begin paragraph admon ---
\paragraph{}
\begin{itemize}
  \item Du kan ikke lese deg til programmering

  \item De fleste synes programmering er krevende i begynnelsen - så blir det utrolig gøy!

  \item Oppskrift på suksess: vær godt forberedt til undervisningen - det gir deg mest fritid og mest læring

  \item Forventet arbeid er 13 timer med INF1100 hver uke \\
    (6 t undervisning, 7 t selvstudium)
\end{itemize}

\noindent
% --- end paragraph admon ---



% !split
\subsection*{Hvor mye matematikk må jeg kunne på forhånd?}


% --- begin paragraph admon ---
\paragraph{}
\begin{itemize}
  \item Nesten alle eksemplene i INF1100 handler om bruk av matematikk

  \item Vi bygger (i prinsippet) på R2 fra vgs

  \item Men matematikken i INF1100 er stort sett \emph{numerisk} matematikk (MAT-INF1100)

  \item Vi håper at INF1100 skal belyse matematikk fra en ny vinkel og hjelpe deg til å forstå matematikk bedre samtidig som du lærer å programmere
\end{itemize}

\noindent
% --- end paragraph admon ---



% !split
\subsection*{Alt undervisningsmateriale er på engelsk}


% --- begin paragraph admon ---
\paragraph{}
\begin{itemize}
  \item Muntlig undervisning foregår på norsk

  \item Alt skriftlig materiale er på engelsk

  \item Hvorfor?

  \item Det mangler gode norske ord for mange ord/uttrykk i programmering

  \item Du finner mye informasjon om programmering på nettet og i bøker - nesten all denne informasjonen er på engelsk og da må du kunne de engelske uttrykkene

  \item Mesteparten av undervisningsmateriellet på UiO er på engelsk

  \item I jobbsammenheng kan du regne med at alt skriftlig foregår på engelsk

  \item Boken og undervisningsmaterialet brukes ved mange utenlandske universiteter
\end{itemize}

\noindent
% --- end paragraph admon ---



% ------------------- end of main content ---------------

% #ifdef PREAMBLE
\end{document}
% #endif

